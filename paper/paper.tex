
\documentclass{statsoc}
\usepackage{graphicx}
\usepackage{listings}
\usepackage{color}
\usepackage{amssymb, amsmath, geometry}
\usepackage{natbib}
\usepackage{hyperref}

\makeatletter
\def\maxwidth{\ifdim\Gin@nat@width>\linewidth\linewidth\else\Gin@nat@width\fi}
\def\maxheight{\ifdim\Gin@nat@height>\textheight\textheight\else\Gin@nat@height\fi}
\makeatother
% Scale images if necessary, so that they will not overflow the page
% margins by default, and it is still possible to overwrite the defaults
% using explicit options in \includegraphics[width, height, ...]{}
\setkeys{Gin}{width=\maxwidth,height=\maxheight,keepaspectratio}

\title[Density ratio estimation for synthetic data]{OUTLINE: Density
ratio estimation as a technique for assessing the utility of synthetic
data}




 
\author[Volker \& Van Kesteren]{Thom Benjamin Volker}
\address{Utrecht University,
Utrecht,
The Netherlands}
\email{t.b.volker@uu.nl}

 
\author[Volker \& Van Kesteren]{Erik-Jan van Kesteren}
\address{Utrecht University,
Utrecht,
The Netherlands}
\email{e.vankesteren@uu.nl}



% BIBLIOGRAPHY
\usepackage[authoryear]{natbib}
\bibpunct{(}{)}{;}{a}{}{,}


% tightlist command for lists without linebreak
\providecommand{\tightlist}{%
  \setlength{\itemsep}{0pt}\setlength{\parskip}{0pt}}




\begin{document}


\begin{abstract}
Abstract goes here
\end{abstract}
\keywords{keywords}

\hypertarget{introduction}{%
\section{Introduction}\label{introduction}}

What is synthetic data?

What is high-utility synthetic data?

\begin{itemize}
\item
  Specific utility: more detailed, but often you do not have sufficient
  knowledge about the analyses that will be performed with the data.
\item
  General utility: provide an intuition of the general quality of the
  synthetic data, and ideally cover the specific utility measures.
\end{itemize}

Current ways to assess the utility?

\begin{itemize}
\item
  pMSE - logistic, regression, CART models (Snoke, Raab, Nowok, Dibben
  \& Slavkovic, 2018; General and specific utility measures for
  synthetic data AND Woo, Reiter, Oganian \& Karr, 2009; Global measures
  of data utlity for microdata masked for disclosure limitation)
\item
  Kullback-Leiber divergence (Karr, Kohnen, Oganian, Reiter \& Sanil,
  2006; A framework for evaluating the utility of data altered to
  protect confidentiality).
\item
  According to multiple authors, both specific and general utility
  measures have important drawbacks (see Drechsler Utility PSD; cites
  others). Narrow measures potentially focus on analyses that are not
  relevant for the end user, and do not generalize to the analyses that
  are relevant. Global utility measures are generally too broad, and
  important deviations in the synthetic data might be missed. Moreover,
  the measures are typically hard to interpret.
\item
  See Drechsler for a paragraph on fit for purpose measures, that lie
  between general and specific utility measures (i.e., plausibility
  checks such as non-negativity; goodness of fit measures as \(\chi^2\)
  for cross-tabulations; Kolmogorov-Smirnov).
\item
  Drechsler also illustrates that the standardized \(pMSE\) has
  substantial flaws, as the results are highly dependent on the model
  used to estimate the propensity scores, and unable to detect important
  differences in the utility for most of the model specifications.
  Hence, it is claimed that a thorough assessment of utility is
  required.
\end{itemize}

Our proposal

\begin{itemize}
\tightlist
\item
  density ratio methods
\end{itemize}

-- Why?

\begin{itemize}
\tightlist
\item
  Implementation in R-package (SynUtility)
\end{itemize}

\hypertarget{density-estimation}{%
\section{density estimation}\label{density-estimation}}

Options: - Prediction models: logistic regression, SVM, CART? -
Multivariate density estimation - density ratio?

Dimension reduction and visualization?

\hypertarget{methodology}{%
\section{Methodology}\label{methodology}}

TO DO

\hypertarget{simulations}{%
\section{Simulations}\label{simulations}}

TO DO

\hypertarget{real-data-example}{%
\section{Real data example}\label{real-data-example}}

TO DO

\hypertarget{results}{%
\section{Results}\label{results}}

TO DO

\hypertarget{discussion-and-conclusion}{%
\section{Discussion and conclusion}\label{discussion-and-conclusion}}

TO DO

\bibliographystyle{rss}
\bibliography{bibliography}


\end{document}
